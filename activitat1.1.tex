% Options for packages loaded elsewhere
\PassOptionsToPackage{unicode}{hyperref}
\PassOptionsToPackage{hyphens}{url}
\documentclass[
]{article}
\usepackage{xcolor}
\usepackage{amsmath,amssymb}
\setcounter{secnumdepth}{-\maxdimen} % remove section numbering
\usepackage{iftex}
\ifPDFTeX
  \usepackage[T1]{fontenc}
  \usepackage[utf8]{inputenc}
  \usepackage{textcomp} % provide euro and other symbols
\else % if luatex or xetex
  \usepackage{unicode-math} % this also loads fontspec
  \defaultfontfeatures{Scale=MatchLowercase}
  \defaultfontfeatures[\rmfamily]{Ligatures=TeX,Scale=1}
\fi
\usepackage{lmodern}
\ifPDFTeX\else
  % xetex/luatex font selection
\fi
% Use upquote if available, for straight quotes in verbatim environments
\IfFileExists{upquote.sty}{\usepackage{upquote}}{}
\IfFileExists{microtype.sty}{% use microtype if available
  \usepackage[]{microtype}
  \UseMicrotypeSet[protrusion]{basicmath} % disable protrusion for tt fonts
}{}
\makeatletter
\@ifundefined{KOMAClassName}{% if non-KOMA class
  \IfFileExists{parskip.sty}{%
    \usepackage{parskip}
  }{% else
    \setlength{\parindent}{0pt}
    \setlength{\parskip}{6pt plus 2pt minus 1pt}}
}{% if KOMA class
  \KOMAoptions{parskip=half}}
\makeatother
\usepackage{longtable,booktabs,array}
\newcounter{none} % for unnumbered tables
\usepackage{calc} % for calculating minipage widths
% Correct order of tables after \paragraph or \subparagraph
\usepackage{etoolbox}
\makeatletter
\patchcmd\longtable{\par}{\if@noskipsec\mbox{}\fi\par}{}{}
\makeatother
% Allow footnotes in longtable head/foot
\IfFileExists{footnotehyper.sty}{\usepackage{footnotehyper}}{\usepackage{footnote}}
\makesavenoteenv{longtable}
\setlength{\emergencystretch}{3em} % prevent overfull lines
\providecommand{\tightlist}{%
  \setlength{\itemsep}{0pt}\setlength{\parskip}{0pt}}
\usepackage{bookmark}
\IfFileExists{xurl.sty}{\usepackage{xurl}}{} % add URL line breaks if available
\urlstyle{same}
\hypersetup{
  hidelinks,
  pdfcreator={LaTeX via pandoc}}

\author{}
\date{}

\begin{document}

\section{Principi 2 -- Principi
d'humilitat}\label{principi-2-principi-dhumilitat}

Aplicat als casos d'ExxonMobil, la transició energètica d'Alemanya i la
transformació d'Ørsted

\subsection{1. Introducció i Context del
Principi}\label{introducciuxf3-i-context-del-principi}

El principi d'humilitat implica la capacitat de qüestionar els
plantejaments propis, escoltar l'evidència científica i actualitzar les
decisions segons el coneixement disponible. En el sector energètic,
aquest principi és especialment rellevant davant la crisi climàtica, la
incertesa tecnològica i els canvis geopolítics.

Aquest anàlisi compara tres casos reals:

\begin{itemize}
\item
  La controvèrsia climàtica d'ExxonMobil
\item
  La política de transició energètica d'Alemanya (Energiewende)
\item
  La transformació corporativa d'Ørsted
\end{itemize}

\subsection{2. Descripció dels Casos}\label{descripciuxf3-dels-casos}

\subsubsection{Cas 1: ExxonMobil (anys
1970--2000)}\label{cas-1-exxonmobil-anys-19702000}

Investigacions internes mostraven que l'empresa coneixia els riscos del
canvi climàtic des dels anys 70-80. Malgrat això, va finançar campanyes
i think tanks que generaven dubte públic sobre l'escalfament global.

🔗
https://en.wikipedia.org/wiki/ExxonMobil\_climate\_change\_controversy

\subsubsection{Cas 2: Energiewende --
Alemanya}\label{cas-2-energiewende-alemanya}

Política energètica iniciada a principis dels anys 2000 amb l'objectiu
de reduir emissions i fomentar renovables. Ha implicat revisions
constants, especialment després de l'accident de Fukushima (2011), que
va accelerar el tancament nuclear.

🔗 https://en.wikipedia.org/wiki/Energiewende

\subsubsection{Cas 3: Transformació
d'Ørsted}\label{cas-3-transformaciuxf3-duxf8rsted}

Ørsted, anteriorment coneguda com DONG Energy (Danish Oil and Natural
Gas), era una empresa centrada en combustibles fòssils. A partir dels
anys 2000 va iniciar una transformació estratègica profunda cap a
l'energia eòlica marina i les renovables, reconeixent la inviabilitat a
llarg termini del model fòssil.

🔗 https://orsted.com/en/about-us/history-and-transformation

Actualment és un dels principals desenvolupadors mundials d'eòlica
offshore.

\subsection{3. Bones Pràctiques: Adaptació i Revisió Crítica (Alemanya i
Ørsted)}\label{bones-pruxe0ctiques-adaptaciuxf3-i-revisiuxf3-cruxedtica-alemanya-i-uxf8rsted}

\subsubsection{Alemanya}\label{alemanya}

Elements d'humilitat institucional:

\begin{itemize}
\item
  Revisió del programa nuclear després de Fukushima
\item
  Increment sostingut de renovables (eòlica i solar)
\item
  Ajustos regulatoris constants
\item
  Participació ciutadana i debat científic
\end{itemize}

Resultat: capacitat d'actualitzar polítiques segons evidència científica
i context geopolític.

\subsubsection{Ørsted}\label{uxf8rsted}

Elements d'humilitat corporativa:

\begin{itemize}
\item
  Reconeixement de la insostenibilitat del model fòssil
\item
  Desinversió progressiva en petroli i gas
\item
  Reorientació total cap a renovables
\item
  Canvi d'identitat corporativa alineat amb objectius climàtics
\item
  Resultat: transformació estratègica basada en l'acceptació del canvi
  científic i regulatori global.
\end{itemize}

\subsection{4. Males Pràctiques: Negació i Resistència al Coneixement
(ExxonMobil)}\label{males-pruxe0ctiques-negaciuxf3-i-resistuxe8ncia-al-coneixement-exxonmobil}

\begin{itemize}
\item
  Finançament de think tanks escèptics
\item
  Comunicació pública contradictòria amb estudis interns
\item
  Retard en l'assumpció de responsabilitat climàtica
\end{itemize}

Impactes:

\begin{itemize}
\item
  Retard en polítiques globals de mitigació
\item
  Desinformació social
\item
  Augment acumulatiu d'emissions
\end{itemize}

\subsection{5. Comparació d'Impactes}\label{comparaciuxf3-dimpactes}

{\def\LTcaptype{none} % do not increment counter
\begin{longtable}[]{@{}
  >{\raggedright\arraybackslash}p{(\linewidth - 4\tabcolsep) * \real{0.1667}}
  >{\raggedright\arraybackslash}p{(\linewidth - 4\tabcolsep) * \real{0.3889}}
  >{\raggedright\arraybackslash}p{(\linewidth - 4\tabcolsep) * \real{0.4444}}@{}}
\toprule\noalign{}
\begin{minipage}[b]{\linewidth}\raggedright
Dimensió
\end{minipage} & \begin{minipage}[b]{\linewidth}\raggedright
Dinamarca (Precaució)
\end{minipage} & \begin{minipage}[b]{\linewidth}\raggedright
Fukushima (No-precaució)
\end{minipage} \\
\midrule\noalign{}
\endhead
\bottomrule\noalign{}
\endlastfoot
Ambiental & Prevenció d'impactes & Contaminació radioactiva \\
Social & Participació pública & Desplaçament massiu \\
Econòmic & Cost preventiu assumible & Cost catastròfic \\
Tecnològic & Avaluació prèvia sistemàtica & Infraestimació de riscos \\
\end{longtable}
}

\subsection{6. Conclusions}\label{conclusions}

El principi d'humilitat exigeix:

\begin{itemize}
\item
  Reconèixer errors i límits del model propi
\item
  Incorporar evidència científica de manera transparent
\item
  Adaptar decisions amb flexibilitat
\end{itemize}

Els casos d'Alemanya i Ørsted demostren que la humilitat pot
convertir-se en avantatge competitiu i institucional, mentre que ignorar
el coneixement científic pot amplificar riscos sistèmics globals.

\section{Principi 3 -- Principi de
precaució}\label{principi-3-principi-de-precauciuxf3}

Aplicat al cas Fukushima i a la planificació eòlica danesa

\subsection{1. Introducció i Context}\label{introducciuxf3-i-context}

El principi de precaució estableix que, davant la possibilitat de danys
greus o irreversibles, cal actuar amb prudència encara que no existeixi
certesa científica absoluta.

En el sector energètic, aquest principi és fonamental en tecnologies
d'alt risc o d'alt impacte territorial.

S'analitzen:

\begin{itemize}
\item
  Accident nuclear de Fukushima Daiichi (2011)
\item
  Planificació eòlica marina a Dinamarca
\end{itemize}

\subsection{2. Descripció dels Casos}\label{descripciuxf3-dels-casos-1}

\subsubsection{Fukushima (2011)}\label{fukushima-2011}

Un terratrèmol i posterior tsunami van provocar la fusió de diversos
reactors nuclears. Investigacions posteriors indiquen que el risc de
tsunami havia estat infraestimat en el disseny i protecció de la
central.

🔗 https://en.wikipedia.org/wiki/Fukushima\_nuclear\_disaster

Impactes:

\begin{itemize}
\item
  Evacuació d'aproximadament 150.000 persones
\item
  Contaminació radioactiva de sòls i aigües
\item
  Cost econòmic superior als 200.000 milions de dòlars
\end{itemize}

\subsubsection{Planificació Eòlica
Danesa}\label{planificaciuxf3-euxf2lica-danesa}

Dinamarca aplica de forma sistemàtica:

\begin{itemize}
\item
  Estudis d'impacte ambiental previs
\item
  Planificació territorial anticipada
\item
  Consultes públiques amb comunitats locals
\item
  Avaluació d'impacte sobre fauna marina
\end{itemize}

🔗 https://ens.dk/en

Aquest model redueix conflictes socials i riscos ambientals.

\subsection{3. Bones Pràctiques: Vigilància Anticipada
(Dinamarca)}\label{bones-pruxe0ctiques-vigiluxe0ncia-anticipada-dinamarca}

Avantatges del model precautori:

\begin{itemize}
\item
  Minimització de l'impacte ecològic
\item
  Major acceptació social
\item
  Reducció de riscos legals i econòmics futurs
\item
  Planificació a llarg termini
\item
  La prevenció esdevé una inversió estratègica.
\end{itemize}

\subsection{4. Males Pràctiques: Subestimació del Risc
(Fukushima)}\label{males-pruxe0ctiques-subestimaciuxf3-del-risc-fukushima}

Factors clau:

\begin{itemize}
\item
  Infraestimació del risc de tsunami
\item
  Insuficient protecció estructural
\item
  Dependència excessiva d'escenaris de risc limitats
\end{itemize}

Conseqüència: una crisi ambiental, social i econòmica de gran escala.

\subsection{5. Comparació}\label{comparaciuxf3}

{\def\LTcaptype{none} % do not increment counter
\begin{longtable}[]{@{}
  >{\raggedright\arraybackslash}p{(\linewidth - 4\tabcolsep) * \real{0.1600}}
  >{\raggedright\arraybackslash}p{(\linewidth - 4\tabcolsep) * \real{0.4133}}
  >{\raggedright\arraybackslash}p{(\linewidth - 4\tabcolsep) * \real{0.4267}}@{}}
\toprule\noalign{}
\begin{minipage}[b]{\linewidth}\raggedright
Dimensió
\end{minipage} & \begin{minipage}[b]{\linewidth}\raggedright
Dinamarca (Precaució)
\end{minipage} & \begin{minipage}[b]{\linewidth}\raggedright
Fukushima (No-precaució)
\end{minipage} \\
\midrule\noalign{}
\endhead
\bottomrule\noalign{}
\endlastfoot
Ambiental & Prevenció d'impactes & Contaminació radioactiva \\
Social & Participació pública & Desplaçament massiu \\
Econòmic & Cost preventiu assumible & Cost catastròfic \\
Tecnològic & Avaluació prèvia sistemàtica & Infraestimació de riscos \\
\end{longtable}
}

\subsection{6. Conclusions}\label{conclusions-1}

El principi de precaució demostra que:

\begin{itemize}
\item
  Prevenir és més eficient que reparar
\item
  La gestió del risc ha de contemplar escenaris extrems
\item
  L'energia segura requereix planificació anticipada
\item
  Els sistemes energètics sostenibles no només han de ser baixos en
  emissions, sinó també resilients i prudents en la seva implementació.
\end{itemize}

\end{document}
